\documentclass[11pt]{article}
\usepackage[margin=0.2in]{geometry}
\usepackage{fullpage}

\begin{document}

\title{DAMN Music}
\author{Group 7: Chang Shen Ho, Hao Liang, Samuel Trew, Akshay Narayan}

\maketitle

\section{Introduction}

\textit{\textbf{Uber for music.}}
\\\\
At the time we wrote our interim checkpoint report we planned to create a simple signal wave generator that converts binary files into wave files. However, after we have finished the oscillator we decided to extend our project to include a random music generator and the ability to generate music with multiple instruments. To achieve this we created our own .damn file format and the DAMN music tracking software.

\section{NO-L Synthesizer}
The NO-L (No Logic) Subtractive Synthesizer is a sound synthesizer based on traditional subtractive synthesizers. A NO-L instrument consists of 3 oscillators, which are superposed to produce a complex, harmony-rich sound.
\subsection{Oscillator}
The oscillator is entirely software based. It is capable of producing frequency-rich waves such as square, saw and triangle waves, in addition to a pure sine wave and random noise. Furthermore, an ADSR envelope is applied to the generated wave to enhance the dynamic expressiveness of the oscillator. The input parameters for an oscillator are an ADSR envelope, a specified wave-shape, the playing frequency and the duration of the note to be played. A constant in use is the sample rate of the program, which is by default set to 44.1kHz (Studio Quality). Bit depth is 16-bit, meaning that the minimum and maximum amplitudes of the waves are -32,768 to 32,767, with 65,536 levels of quantization. 
\\\\
An ADSR envelope stands for an Attack-Decay-Sustain-Release envelope. These values describe the limited amplitude of a wave at certain points in the wave. The sustain level describes the percentage of maximum amplitude the sustain phase is held at. The attack, decay and release values are described in milliseconds, and describe the time taken to change amplitude. The attack phase is a change from 0\% to 100\%, the decay from 100\% to the sustain level, and the release from the sustain level to 0\%.
\subsection{Instrument}
A NO-L instrument consists of 3 oscillators. Each oscillator has its individual parameters, in addition to a volume mix level, an octave offset and a detune offset. The instrument has parameters for each oscillator, the note to be played, the duration to play it for and the velocity of the note.
\\\\
The note is given in scientific pitch notation, with C4 (middle C) being designated 60. The frequency of the note is calculated using the equation: 
\[f_n=f_0\times 2^{n/12}\]
The variable \(n\) is the distance of the note in semitones from concert A (A4), and the constant \(f_0\) is the frequency of A4, which is set as 440Hz. An octave offset changes the number of semitones by 12 (changes octaves), and the detune offset changes it in increments of 0.01 semitones (cents).\\\\
The velocity of a note is essentially its relative volume in the track.

\section{DAMN Music Tracking Software}
DAMN stands for Digitally Approximated Music using NO-L. It is inspired by the concept of a music tracker, where notes are laid out on separate channels on discrete frames on a music time-line. The DAMN tracker supports multiple instruments (polyphony), and discrete note lengths and velocities for each note. The sound produced by DAMN is a 16-bit PCM wave file, with a 44.1kHz sample rate and a single channel.
\subsection{DAMN File Format}
The .damn file format is a proprietary file format that is read by the DAMN tracking software. It has a 16-byte header, containing information about the tempo of the piece, the number of instruments and the frames per beat. Starting from address \texttt{0x10} are the instrument settings. Each instrument is a 48-byte block, with each oscillator taking up 16 bytes. This carries all the parameters required for the instrument to produce sound.\\\\
The remainder of the file is tracking information for the DAMN tracker to synthesize. A frame is a block of memory \texttt{4 * number of instruments} bytes large. A note for a single instrument is 4 bytes large, and holds information about pitch, velocity, and duration. Pitch is represented as in NO-L, velocity is an unsigned integer from 0-255, and sustain is an unsigned integer from 0-65,536, representing duration in milliseconds. If the duration is set as 0, the software will interpret this as a held note, to be released when another non-zero note is read, or until a stop signal. The stop signal is when the pitch is 255 (\texttt{0xFF}), and when it is called no note will be played and the previous note will stop. If the velocity of a note is 0, the note is ignored.\\\\
The file is ended with a 4-byte series of 1's (\texttt{0xFFFFFFFF}). After this signal is received, the rest of the file will be ignored. More information on the .damn file format can be found in the appendix.

\subsection{Synthesis}
The DAMN tracker first initializes an array of \texttt{uint16\_t} using \texttt{calloc}. The maximum size of the array is limited by the maximum value of \texttt{unsigned long long}, which is at least \(1.84\times 10^{19}\). Using a sample rate of 44.1kHz, this means a maximum length of \(4.18\times 10^{14}\) seconds, or 13.3 million years.
\\\\
While the tracker reads data from the file, it keeps a frame pointer, and a set of arrays used for generating sustained notes. When a note of duration 0 is encountered, its values are put into the sustain arrays. Every subsequent read which returns zeroes increments the sustain duration counter by one. When a note is encountered, the tracker calls the synthesizer, with the recorded duration, and superposes the wave generated at the position where the note was encounterd. If a note has a non-zero duration, it is immediately generated and superposed at the current frame position. The frame increments everytime it has read \(n\) notes, where \(n\) is the number of instruments.

\subsection{Writing to WAV}
We opted to write to a Microsoft WAV file as the Linear PCM format means it requires the least amount of processing. All that is required is a WAVE header containing the characters "RIFF WAVE", Subchunk1 containing the audio format, the number of channels, sample rate and bit-depth, and Subchunk2 containing all the wave data.
\\\\
The WAV is represented internally as a struct containing all the components of a WAV. The generated wave data is written to the struct, and then the entire struct is written sequentially to a file.

\subsection{Randomiser}
We ended up taking two different approaches with our randomiser. We started out creating a very basic software which contained 15 whole notes and the music generated was not allowed to go out of these bounds. 
\\
Using our musical knowledge we created probabilities based on how likely it would be for the next note to play based on the previous note. This was done with arpeggios in mind so that the transitions were deemed "musical" at a most fundamental level. Then using scientific pitch notation we converted our numbers created into sound values in binary.
\\
However, this caused issues with semitones and transition between notes such as E and F due to the spacing between whole notes and arpeggio notes varying. We were able to create a piece of custom length with up to 3 instruments with variable types of waves (sin, triangular and square) but it was never the most musical of pieces.
\\\\
We then precoded to use our musical knowledge to create a much more complex and harmonious piece of music. This version took into account the sharps and flats of music and added a wider range in which the notes could compile

\section{Group Reflection}

Do Not Over-clock Pi to 1GHz, corrupted gcc.
\\\\
Delete + reclone = working repo

\section{Individual Reflections}
\subsection{Chang Shen Ho}
Writing the assembler, oscillator, NO-L and DAMN taught me a lot about C, like how to manipulate memory pointers and using file I/O. Writing the DAMN interface using the curses library also taught me basic GUI building skills.\\\\
One of the main problems I encontered was \texttt{Segmentation Fault} occuring everywhere, and the occasional \texttt{Stack Smashing Detected}. This has taught me how to properly allocate memory and strings, and through debugging using Valgrind, how to properly free allocated memory.\\\\
One thing I would like to continue working on is the GUI for 

\subsection{Hao Liang}
I am not the most proficient with C in the group but I did a fair amount of the emulator and helped in debugging and cleaning up the code. I made the mistake of not re-basing back to master after I merged my branch during emulator, which caused the subsequent merge to be very complicated since we heavily reformatted our code after that. I communicated well with my team and we worked together well in labs. On the next group project I would utilise branches properly to separate the work-flow with other group members.
\subsection{Samuel Trew}
My confidence in C was never the highest but I implemented the Branch instruction in Emulator and worked on the 


\subsection{Akshay Narayan}
Unfortunately it took me much longer to understand the problems that we had to solve than other members of the group, and as such I did not actually make a start on the code as quickly. I was responsible for implementing the single data transfer instruction in the emulator, and the main reason I struggled with this was that I was unsure on the order in which bits in the instruction were to be evaluated. I think that my understanding of architecture generally was not up to scratch, which did not help. In terms of the extension, I helped set up the representation of a musical note in C. I communicated well with the rest of the team and worked well together in labs. On the next group project I would undertake extensive wider reading to ensure that I understand the essence of the problems we are facing.\\\\

\pagebreak

\section*{Appendix}
\subsection*{DAMN File Format}
The DAMN Header is detailed in Table 1.
\begin{table}[]
\centering
\caption{The DAMN header format}
\label{my-label}
\begin{tabular}{|c|c|c|c|c|c|c|c|c|}
\hline
\textbf{Offset} & \textit{0x0} & \textit{0x1} & \textit{0x2} & \textit{0x3} & \textit{0x4} & \textit{0x5} & \textit{0x6} & \textit{0x7} \\ \hline
\textit{0x00}   & 'D'          & 'A'          & 'M'          & 'N'          & \multicolumn{4}{c|}{Number of Instruments}                \\ \hline
\textit{0x08}   & Tempo        & Frames/Beat  & \multicolumn{6}{c|}{\textit{Unused}}                                                    \\ \hline
\textit{0x10}   & \multicolumn{8}{c|}{\textit{Instrument Settings Begin Here}}                                                          \\ \hline
\end{tabular}
\end{table}

\end{document}

